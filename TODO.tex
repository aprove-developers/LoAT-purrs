\documentclass[a4paper]{article}
\usepackage{amsmath}

\begin{document}

\begin{enumerate}

\item (Added on 4.9.2002)
In \texttt{simplify.cc}, emphasize the fact that $n$  is an integer variable:
for example, the expression $(1-(-1)^n) \cdot (1+(-1)^n)$ simplifies to 
0 only if n is an integer.

\item (Added on 4.9.2002)
Use \texttt{numer\_denom()} as a default simplification of expressions?
Consider the example
\[
  \begin{aligned}
  &-(1-a^{-1})^{-1}*a^{-1}
  -
  3*(1-a^{-1})^{-2}*a^{-2}
  +
  (1-a^{-1})^{-2}*a^{-1} \\
  &+
  2*(1-a^{-1})^{-3}*a^{-2}
  -
  2*(1-a^{-1})^{-3}*a^{-3},
  \end{aligned}
\]
which \texttt{GiNaC} does not simplify. After calling 
\texttt{numer\_denom()} on this
expression, we get the list $\{\, 0$, $1\,\}$.
This arises from the recurrence $x_n = a*x_{n-1} + n^2$ in the
file \texttt{recurrences}.
FIXED.

\item (Added on 4.9.2002)
\texttt{purrs} fails to check that Binet's formula holds for the Fibonacci
sequence because it does not simplify the formula as much as it should.
Consider the possibility of introducing formal parameters to perform the
simplifications on expressions that do not contain irrational quantities.

\item (Added on 4.9.2002)
Give conditions on parameters when solving parametric recurrences.

\item (Added on 16.9.2002)
Find suitable notations in \texttt{GiNaC} for formal sums, formal products
and hypergeometric functions.
Also, give rules to combine and simplify sums and products.

\item (Added on 20.9.2002)
In order to apply the ``transcendental'' methods of summation contained
in the book by Petkov\v sek, Wilf and Zeilberger, we
need to be able to simplify ratios of factorials.
Of course the same remark holds for binomial coefficients.
\texttt{GiNaC} shows no ability at that (just try
\texttt{factorial$(n+1)$/factorial$(n)$}).

\item (Added on 9.10.2002)
We have the following problem with Gosper's algorithm, in particular when
we compute the resultant's roots:
given the recurrence
\[
  x_n = 2 \cdot x_{n-1} - x_{n-2} + \frac1{n(n+1)}
\]
we use the formula 
\[
  x_n = \sum_{i=k}^n g_{n-i} p(i)
  +
  \sum_{i=0}^{k-1} g_{n-i} \Bigl( x_i - \sum_{j=1}^i a_j x_{i-j} \Bigr).
\]
We consider the first term of the previous formula where, in this case,
we have $g(n) = n + 1$, then $g(n-k) = n - k + 1$, and
$p(k) = \frac1{k(k+1)}$, then the sum to compute is
\[
  \sum_{k=2}^n \frac{n-k+1}{k(k+1)}.
\]
We consider the steps of Gosper's algorithm
(for clarity we change the variables):
\begin{align*}
  t_n &= \frac{a-n+1}{n(n+1)},\\
  r_n &= \frac{t_{n+1}}{t_n} = \frac{n(a-n)}{(1-n+a)(n+2)}.
\end{align*}
In the second step we have $f(n) = an - n^2$ e
$g(n) = -n^2 + an - n + 2 + 2a$, then
$g(n+h) = -(h+n)^2 + a(n+h) - n - h + 2 + 2a$.
The resultant is
\[
  R = 4 + 2a^2 + h^4 - 3h^2 + 2h^3 -4h -3ah - ha^2 -3h^2a -h^2a^2 + 6a
\]
and the constant term
\[
  4 + 2a^2 + 6a.
\]
How we compute the roots for the resultant is not right in this case.

\item (Added on 10.10.2002)
Modify \texttt{simplify.cc} so that only one call to the
function \texttt{simplify\_on\_output\_ex()} is necessary to arrive
at the maximum simplification of the expression in input. 
For example, consider the recurrence:
\[
  x(n) = (n-2)^2+x(n-2);
\]
when we try to verify the solution of the recurrence, we have to compute
the difference $x(n) - (n-2)^2 - x(n-2)$, which is simplified to  
$5/2+((-1)^n)^2*n-n-5/2*((-1)^n)^2$, and not to $0$.
We remark that the expression $5/2+((-1)^n)^2*n-n-5/2*((-1)^n)^2$ is
simplified to $0$ on a further call to \texttt{simplify\_on\_output\_ex()}.

\end{enumerate}

\end{document}
