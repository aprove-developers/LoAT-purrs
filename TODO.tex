\documentclass[a4paper]{article}
\usepackage{amsmath}

\begin{document}

\begin{enumerate}

\item (Added on 4.9.2002)
In \texttt{simplify.cc}, emphasize the fact that $n$  is an integer variable:
for example, the expression $(1-(-1)^n) \cdot (1+(-1)^n)$ simplifies to 
0 only if n is an integer.

\item (Added on 4.9.2002)
Use \texttt{numer\_denom()} as a default simplification of expressions?
Consider the example
\[
  \begin{aligned}
  &-(1-a^{-1})^{-1}*a^{-1}
  -
  3*(1-a^{-1})^{-2}*a^{-2}
  +
  (1-a^{-1})^{-2}*a^{-1} \\
  &+
  2*(1-a^{-1})^{-3}*a^{-2}
  -
  2*(1-a^{-1})^{-3}*a^{-3},
  \end{aligned}
\]
which \texttt{GiNaC} does not simplify. After calling 
\texttt{numer\_denom()} on this
expression, we get the list $\{\, 0$, $1\,\}$.
This arises from the recurrence $x_n = a*x_{n-1} + n^2$ in the
file \texttt{recurrences}.

\item (Added on 4.9.2002)
\texttt{purrs} fails to check that Binet's formula holds for the Fibonacci
sequence because it does not simplify the formula as much as it should.
Consider the possibility of introducing formal parameters to perform the
simplifications on expressions that do not contain irrational quantities.

\item (Added on 4.9.2002)
Give conditions on parameters when solving parametric recurrences.

\end{enumerate}

\end{document}
