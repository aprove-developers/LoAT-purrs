\documentclass[a4paper]{article}
\usepackage{amsmath}

\begin{document}

\begin{enumerate}

\item (Added on 4.9.2002)
In \texttt{simplify.cc}, emphasize the fact that $n$  is an integer variable:
for example, the expression $(1-(-1)^n) \cdot (1+(-1)^n)$ simplifies to
0 only if n is an integer. FIXED.

\item (Added on 4.9.2002)
Use \texttt{numer\_denom()} as a default simplification of expressions?
Consider the example
\[
  \begin{aligned}
  &-(1-a^{-1})^{-1}*a^{-1}
  -
  3*(1-a^{-1})^{-2}*a^{-2}
  +
  (1-a^{-1})^{-2}*a^{-1} \\
  &+
  2*(1-a^{-1})^{-3}*a^{-2}
  -
  2*(1-a^{-1})^{-3}*a^{-3},
  \end{aligned}
\]
which \texttt{GiNaC} does not simplify. After calling
\texttt{numer\_denom()} on this
expression, we get the list $\{\, 0$, $1\,\}$.
This arises from the recurrence $x_n = a*x_{n-1} + n^2$ in the
file \texttt{recurrences}.
FIXED.

\item (Added on 4.9.2002)
\texttt{purrs} fails to check that Binet's formula holds for the Fibonacci
sequence because it does not simplify the formula as much as it should.
Consider the possibility of introducing formal parameters to perform the
simplifications on expressions that do not contain irrational quantities.
FIXED.

\item (Added on 4.9.2002)
Give conditions on parameters when solving parametric recurrences.

\item (Added on 16.9.2002)
Find suitable notations in \texttt{GiNaC} for formal sums, formal products
and hypergeometric functions.
Also, give rules to combine and simplify sums and products.
FIXED formal sums and formal product.

\item (Added on 20.9.2002)
In order to apply the ``transcendental'' methods of summation contained
in the book by Petkov\v sek, Wilf and Zeilberger, we
need to be able to simplify ratios of factorials.
Of course the same remark holds for binomial coefficients.
\texttt{GiNaC} shows no ability at that (just try
\texttt{factorial$(n+1)$/factorial$(n)$}).
FIXED factorials.

\item (Added on 9.10.2002)
We have the following problem with Gosper's algorithm, in particular when
we compute the resultant's roots:
given the recurrence
\[
  x_n = 2 \cdot x_{n-1} - x_{n-2} + \frac1{n(n+1)}
\]
we use the formula
\[
  x_n = \sum_{i=k}^n g_{n-i} p(i)
  +
  \sum_{i=0}^{k-1} g_{n-i} \Bigl( x_i - \sum_{j=1}^i a_j x_{i-j} \Bigr).
\]
We consider the first term of the previous formula where, in this case,
we have $g(n) = n + 1$, then $g(n-k) = n - k + 1$, and
$p(k) = \frac1{k(k+1)}$, then the sum to compute is
\[
  \sum_{k=2}^n \frac{n-k+1}{k(k+1)}.
\]
We consider the steps of Gosper's algorithm
(for clarity we change the variables):
\begin{align*}
  t_n &= \frac{a-n+1}{n(n+1)},\\
  r_n &= \frac{t_{n+1}}{t_n} = \frac{n(a-n)}{(1-n+a)(n+2)}.
\end{align*}
In the second step we have $f(n) = an - n^2$ e
$g(n) = -n^2 + an - n + 2 + 2a$, then
$g(n+h) = -(h+n)^2 + a(n+h) - n - h + 2 + 2a$.
The resultant is
\[
  R = 4 + 2a^2 + h^4 - 3h^2 + 2h^3 -4h -3ah - ha^2 -3h^2a -h^2a^2 + 6a
\]
and the constant term
\[
  4 + 2a^2 + 6a.
\]
How we compute the roots for the resultant is not right in this case.

\item (Added on 10.10.2002)
Modify \texttt{simplify.cc} so that only one call to the
function \texttt{simplify\_on\_output\_ex()} is necessary to arrive
at the maximum simplification of the expression in input.
For example, consider the recurrence:
\[
  x(n) = (n-2)^2+x(n-2);
\]
when we try to verify the solution of the recurrence, we have to compute
the difference $x(n) - (n-2)^2 - x(n-2)$, which is simplified to
$5/2+((-1)^n)^2*n-n-5/2*((-1)^n)^2$, and not to $0$.
We remark that the expression $5/2+((-1)^n)^2*n-n-5/2*((-1)^n)^2$ is
simplified to $0$ on a further call to \texttt{simplify\_on\_output\_ex()}.

\item (Added on 24.10.2002)
Write a message to ginac list about the problem of initialization mechanism
of the symbolic functions (x(n), sum(), ...).

\item (Added on 6.1.2003)
It is necessary for several reasons (efficiency, approximation,
readability, \dots) to simplify as much as possible the solution.
I collect some remarks on one particular case (Fibonacci strikes
again), but they apply generally to linear recurrences with constant
coefficients.
I tried $x(n) = x(n-1) + x(n-2) + 2n$ and the (correct) answer is
%
\begin{multline*}
  2 (\psi-\phi)^{-1} \phi^n
  -
  2 \psi^n (\psi-\phi)^{-1} \\
  +
  2 (-1+\phi)^{-2} (\psi-\phi)^{-1} \phi
  -
  2 (-\psi \phi+\psi^2
  -
  \psi+\phi)^{-1} \psi n \\
  +
  (-(\psi-\phi)^{-1} \phi^n
  +
  \psi^n (\psi-\phi)^{-1}) x(1) \\
  +
  2 (-\phi^2+\psi \phi-\psi+\phi)^{-1} n \phi
  -
  2 (-1+\phi)^{-2} (\psi-\phi)^{-1} \phi^{2+n} \\
  +
  2 (-1+\psi)^{-2} \psi^{2+n} (\psi-\phi)^{-1}
  -
  2 (-\psi \phi+\psi^2-\psi+\phi)^{-1} \psi \\
  +
  (-(\psi-\phi)^{-1} \phi^{1+n}
  +
  (\psi-\phi)^{-1} \phi^n \\
  -
  \psi^n (\psi-\phi)^{-1}
  +
  \psi^{1+n} (\psi-\phi)^{-1}) x(0) \\
  +
  2 (-\phi^2+\psi \phi-\psi+\phi)^{-1} \phi
  -
  2 (-1+\psi)^{-2} (\psi-\phi)^{-1} \psi,
\end{multline*}
where $\phi$ and $\psi$ stand for the two roots of the characteristic
equation $\lambda^2 = \lambda + 1$.
We see a proliferation of \emph{constant} coefficients (coefficients
that depend only on $\phi$ and $\psi$, not on $n$), some repeated many
times.

If we collect the term $A := (\psi-\phi)^{-1}$, we can simplify the answer
a little bit:
\begin{multline*}
  2 A (\phi^n - \psi^n) + 2 (-1+\phi)^{-2} A \phi
  -
  2 (-\psi \phi+\psi^2 - \psi+\phi)^{-1} \psi n \\
  +
  A (-\phi^n + \psi^n) x(1) \\
  +
  2 (-\phi^2+\psi \phi-\psi+\phi)^{-1} n \phi
  -
  2 (-1+\phi)^{-2} A \phi^{2+n} \\
  +
  2 (-1+\psi)^{-2} \psi^{2+n} A
  -
  2 (-\psi \phi+\psi^2-\psi+\phi)^{-1} \psi \\
  +
  A (- \phi^{1+n} + \phi^n - \psi^n + \psi^{1+n}) x(0) \\
  +
  2 (-\phi^2+\psi \phi-\psi+\phi)^{-1} \phi
  -
  2 (-1+\psi)^{-2} A \psi
\end{multline*}
The following table collects some identities satisfied by $\phi$ and
$\psi$: in particular, since they satisfy a polynomial of degree 2, it
is \emph{always possible} to replace $\phi^2$, say, by an expression
containing only $\phi$ (in this particular case $\phi^2 = \phi + 1$).
\begin{align*}
  (\psi-\phi)^{-1} &= A      \\
  \phi \psi        &= - 1    \\
  \phi^2 - \phi    &= 1      \\
  \psi^2 - \psi    &= 1      \\
  (\psi - 1)^{-1}  &= \psi   \\
  (\phi - 1)^{-1}  &= \phi   \\
  (\psi - 1)^{-2}  &= \psi^2 \\
  (\phi - 1)^{-2}  &= \phi^2
\end{align*}
Using these identities repeatedly, we obtain the simpler form
\begin{multline*}
  2 A (\phi^n - \psi^n) + 2 \phi^3 A
  -
  2 (2 + \phi)^{-1} \psi n
  +
  A (-\phi^n + \psi^n) x(1) \\
  +
  2 (- 2 - \psi)^{-1} n \phi
  -
  2 A \phi^{4+n}
  +
  2 A \psi^{4+n} \\
  -
  2 (2 + \phi)^{-1} \psi
  +
  A (- \phi^{1+n} + \phi^n - \psi^n + \psi^{1+n}) x(0)
  +
  2 (- 2 - \psi)^{-1} \phi
  -
  2 \psi^3 A
\end{multline*}
Now one should compute the actual values of $A$, $\psi^3$, $\phi^3$,
$(2 + \phi)^{-1}$ and $(- 2 -\psi)^{-1}$, and this will yield an even
simpler form.
Probably, the best place for these simplification is at the earliest
possible moment when the expressions involving $\phi^2$, \dots, arise.

\item (Added on 19.1.2003)
Complete the part dealing with finite-order recurrences with variable
coefficients.

\item (Added on 19.1.2003)
Upper and lower bounds for exact solutions, when they are written as
symbolic sums or symbolic products.
Examples: $x(n) = x(n-1) + 1/n$, $x(n) = (3n+1) x(n-1)$, \dots

\item (Added on 8-3-2003)
To check if it is possible to improve the part of the computation of the
symbolic sums (when the inhomogeneous term is polynomial, exponential or
product of them) removing the two vectors \texttt{symbolic\_sum\_distinct} and
\texttt{symbolic\_sum\_no\_distinct}.
Now the Gosper's algorithm is applied only when the order of the recurrences
with constant coefficients is $1$: extend it!

\item (Added on 8-3-2003)
In the function \texttt{solve\_variable\_coeff\_order\_1()} before use of the
Gosper's algorithm to try to use the method for polynomials and
exponentials (maybe this method is more efficient than the
Gosper's algorithm?)

\item (Added on 9-3-2003)
In the function \texttt{find\_roots()} there is a \texttt{switch} on the
order of the equation but at this point of the function is impossible
that the degree is $1$: in fact we consider equations with integer
coefficients and we have already found rational roots.
Therefore, the \texttt{case 1} of the \texttt{switch} is useless.
Is useless also the function \texttt{nested\_polynomial()} because
we already applied the order reduction.

\item (Added on 1-4-2003)
Fix the problem of initial conditions (to find the index starting from which
the recurrence is well-defined).

\item (Added on 3-4-2003)
Take the argument of the function \verb/sum/ or \verb/prod/ and try to
compute the sum with the functions \verb/sum_poly_times_exponentials()/
or \verb/full_gosper()/ and the product with the function
\verb/compute_prod()/.

\item (Added on 27-6-2003)
Decide whether the degree of a polynomial is a \verb/Number/ (as in
the old definition of \verb/Polynomial_Root/) or an \verb/unsigned/
(as in \verb/Expr::degree()/).
We have temporarily standardized to \verb/unsigned/, but the issue
deserves further study.

\item (Added on 8.4.2005)
\emph{Verification of the solution}: too many recurrences have a
solution that \texttt{purrs} does not verify.
These include linear recurrences of finite order with constant
coefficients whose characteristic equation has irrational roots, and
many ``infinite order'' recurrences.

\item (Added on 8.4.2005)
\emph{Verification of the solution}:
when the recurrence is rewritten, check the correctness of the
solution of the \emph{original} recurrence.

\item (Added on 8.4.2005)
\emph{Verification of the solution}:
write a test for the verification procedure.

\item (Added on 8.4.2005)
\emph{Verification of the solution}:
Consider the possibility of \emph{numerical} tests of the solution and
of its approximations.

\item (Added on 8.4.2005)
\emph{Linear recurrences of finite order with constant coefficients}:
Use the function \verb/sum_poly_alt()/ to compute the symbolic sum of
polynomials.

\item (Added on 8.4.2005)
\emph{Linear recurrences of finite order with constant coefficients}:
Choose a syntax for systems, and write a filter to handle them.

\item (Added on 8.4.2005)
\emph{Linear recurrences of finite order with constant coefficients}:
Handle symbolic sums arising from summands in the non-homogeneous term
that are not linear combinations of products of polynomials and
exponentials.
When possible, express these sums in terms of higher transcendental
functions, or as partial sums of hypegeometric series.

\item (Added on 8.4.2005)
\emph{Infinite order recurrences}:
Extend both code and documentation to cover the more general case
\[
  x_n
  =
  f(n) \sum_{i = a}^{n-b} x_i + g(n),
\]
where $a$ and $b$ are fixed non-negative integers.

\item (Added on 8.4.2005)
\emph{Divide-et-impera recurrences}:
Deal with recurrences of the type $x(n) = \alpha x(n/\beta) + g(n)$
where $\beta > 1$ is \emph{not} an integer.

\item (Added on 8.4.2005)
\emph{Divide-et-impera recurrences}:
Deal with recurrences of the type $x(n) = \alpha(n) x(n/\beta) + g(n)$
where $\alpha(n)$ is a function.

\item (Added on 8.4.2005)
\emph{Divide-et-impera recurrences}:
Deal with recurrences of rank 2 or more.

\item (Added on 8.4.2005)
\emph{Linear recurrences with variable coefficients of order $1$}:
allow parameters in the variable coefficient.

\item (Added on 8.4.2005)
\emph{Linear recurrences with variable coefficients of order $1$}:
Express the solution in terms of Euler's Gamma function, of the
incomplete Gamma function and other higher transcendental functions,
including hypergeometric functions.

\item (Added on 8.4.2005)
\emph{Linear recurrences with variable coefficients of any order}:
Complete the implementation of Zeilberger's method, and other
algorithms deriving from holonomy theory.

\item (Added on 8.4.2005)
\emph{Linear recurrences with variable coefficients of any order}:
Implement the \emph{generating function} method.

\item (Added on 8.4.2005)
\emph{Approximation of the solution, given the exact solution}:
Fully implement the method described in valid.tex.

\item (Added on 8.4.2005)
\emph{Approximation of the solution, given the exact solution}:
Approximations arising from partial sums of suitable Taylor series.
See approximations.tex.

\item (Added on 8.4.2005)
\emph{Approximation of the solution, without computing it explicitly}:
See approximations.tex.

\item (Added on 8.4.2005)
\emph{Multivariate recurrences}.

\item (Added on 8.4.2005)
\emph{Special recurrences}:
Deal with recurrences arising from Newton's algorithm.

\end{enumerate}


\end{document}
