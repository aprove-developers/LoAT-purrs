\documentclass{article}[12]
\textwidth=16.0cm
\textheight=25cm
\topmargin=-1.5cm
\oddsidemargin=1cm
\evensidemargin=1cm

\begin{document}

\def\dx{{\rm d}}
\def\N{\mathbf N}
\def\Z{\mathbf Z}
\def\R{\mathbf R}
\def\C{\mathbf C}
\def\({\bigl(}
\def\){\bigr)}

\centerline{Facciamo un po' il punto \dots}
\bigskip
\noindent
Mi sembra utile fare il punto della situazione, anche per sapere
quello che sappiamo fare e 
quello che vorremmo saper fare. Naturalmente l'intersezione di questo
messaggio con i 
precedenti non \`e vuota ed \`e utile averli presenti. 

Segnalo subito una cosa che non ho capito bene: \`e pi\'u utile
cercare procedimenti molto 
generali che si applicano ad una grande variet\`a di casi, oppure \`e
meglio cercare soluzioni 
dirette e semplici in alcuni casi presumibilmente frequenti e
ricorrere alla teoria generale 
solo nei casi pi\'u difficili? 

\bigskip
\leftline{\bf Ricorrenze lineari di ordine finito a coefficienti
costanti}
\bigskip\noindent
Per noi sono ricorrenze del tipo 
$$
x_n = a_1 x_{n-1} + a_2 x_{n-2} + \cdots + a_k x_{n-k} + f(n)
\eqno(1)
$$
in cui $k$ \`e un intero positivo fissato che chiamiamo {\it ordine
della ricorrenza\/} 
(implicitamente supponiamo che $a_k\ne0$, altrimenti l'ordine \`e
pi\'u piccolo), gli $a_j$ 
sono numeri reali o complessi, ed $f$ \`e una funzione definita sui
numeri naturali. 
Naturalmente una ricorrenza di questo tipo \`e data insieme ai valori
$x_0$, $x_1$, \dots, $x_{k-1}$, 
che supponiamo noti. Se $f\equiv0$, si dice che la ricorrenza \`e {\it
omogenea}. 
L'equazione omogenea associata \`e per l'appunto l'equazione  in cui
si prende $f\equiv0$, e cio\`e, 
per distinguerla dalla precedente scriveremo 
$$
g_n = a_1 g_{n-1} + a_2 g_{n-2} + \cdots + a_k g_{n-k}.
\eqno(2)
$$
Questa deve comunque essere risolta per prima con il metodo dell'{\it
equazione caratteristica\/} 
che, per definizione, \`e l'equazione 
$$
\lambda^k = a_1\lambda^{k-1} + \cdots + a_{k-1}\lambda + a_k.
$$
In generale, questa ha $k$ radici complesse $\lambda_1$, \dots,
$\lambda_k$. (Segnalo un problema: 
se $k\ge5$ non esiste la formula risolutiva generale di questa
equazione; se $k=3$ o $k=4$ esiste 
ma \`e molto incasinata). Se queste sono tutte distinte, la soluzione
generale della ricorrenza 
omogenea \`e 
$$
g_n = \alpha_1 \lambda_1^n + \cdots + \alpha_k \lambda_k^n.
\eqno(3)
$$
(Se ci sono due o pi\'u soluzioni coincidenti (per inciso, non \`e
difficile verificare con GiNaC 
se questo \`e il caso), la (3) deve essere opportunamente modificata).

Questo significa che {\it ogni\/} successione del tipo (2) ha una
soluzione del tipo (3), per 
opportuni valori dei coefficienti $\alpha_j$, che devono essere
determinati a partire dai valori 
iniziali $g_0$, \dots $g_{k-1}$. Il problema \`e che, anche se i
numeri $g_n$ sono tutti interi, 
non \`e affatto detto che anche i $\lambda$ debbano essere interi:
possono essere irrazionali (vedi 
Fibonacci) o addirittura complessi. Presumibilmente nei casi
interessanti per la tesi di Tatiana 
i numeri complessi non dovrebbero comparire, ma non \`e difficile
immaginare problemi combinatori 
in cui compaiono i numeri di Fibonacci e quindi soluzioni irrazionali:
come \`e possibile gestire 
questi casi con GiNaC? 

Una volta trovata la soluzione generale dell'equazione omogenea (2)
dobbiamo risolvere l'equazione 
completa (1): per linearit\`a \`e sufficiente trovare {\it una\/}
qualunque soluzione dell'equazione 
completa, sommare la soluzione generale dell'omogenea ed imporre le
condizioni iniziali. 
Illustro i problemi per mezzo di un esempio: se l'equazione da
risolvere \`e la pi\'u semplice 
possibile 
$$
x_n = x_{n-1} + f(n)
\eqno(4)
$$
la procedura precedente d\`a $g_n=\alpha_1\cdot1^n=\alpha_1$. Osservo
che la soluzione generale 
di (4) \`e infatti 
$$
x_n = x_0 + f(1) + f(2) + \cdots + f(n).
$$
($g_n=\alpha_1$ corrisponde alla costante $x_0$). Per trovare una
formula chiusa abbiamo bisogno di 
due cose: 
\begin{itemize}
\item formule esplicite (o algoritmi) per il calcolo esatto nei casi
pi\'u frequenti; 
\item formule di sommazione (vedi Appendice in calce) per i casi che
rimangono fuori. 
\end{itemize} 
Per esempio, se $f(n)={1\over n}$ allora 
$$
x_n = x_0 + \sum_{k=1}^n {1\over k}
$$
e non esiste una formula chiusa esplicita per questa somma, mentre le
formule di sommazione qui 
sotto danno entrambe 
$$
x_n = x_0 + \log n + \gamma + O\( n^{-1} \)
$$
(cfr il mio messaggio del 14.12), dove $\gamma\approx0.577215\dots$
\`e la costante di Eulero. 
Le stesse formule forniscono anche le disuguaglianze esplicite 
$$
0\le x_n - \( x_0 + \log n \) \le 1.
$$
Domanda: questo \`e di qualche utilit\`a? L'uso automatico di
generiche formule di sommazione 
mi sembra piuttosto complesso: forse \`e utile individuare dei casi
interessanti o frequenti. 

Per quello che riguarda le formule esplicite (o algoritmi) per il
calcolo esatto, ho in precedenza 
affermato che si pu\`o trovare in generale la formula chiusa se $f(n)$
\`e una combinazione 
lineare di polinomi ed esponenziali: in effetti vale il seguente 

\noindent{\bf Lemma. }{\sl Dato $p\in\Z[n]$ di grado $d$ esistono
interi $b_0$, \dots, $b_d$ 
tali che 
$$
p(n) = \sum_{j=0}^d b_j {n\choose j}
\qquad{\rm dove}\qquad
{n\choose j} = {n(n-1)\cdots(n-j+1)\over j!}.
$$}
L'algoritmo per il calcolo dei coefficienti richiede $d$ iterazioni:
infatti se $p(n)=a_d n^d+\cdots+a_0$ 
dove $a_d\ne0$, allora $b_d=a_d\cdot d!$ e quindi si pu\`o continuare
con $p(n)-b_d{n\choose d}$ 
che ha grado $\le d-1$. Di questo ho gi\`a parlato con Tatiana, e mi
garantisce che con GiNaC non 
\`e un problema realizzare l'algoritmo relativo. Una volta scritto $p$
in questa forma, possiamo 
applicare l'algoritmo che ho descritto una volta alla lavagna: dato
che conosciamo esplicitamente 
$$
\sum_{j=0}^n x^j = {1-x^{n+1}\over 1-x}
\eqno(5)
$$
(per il momento suppongo $x\ne1$), abbiamo, per esempio 
$$
\sum_{j=k}^n j(j-1)\cdots(j-k+1) x^j
=
x^k {\dx^k\over \dx x^k} \sum_{j=k}^n x^j
=
x^k {\dx^k\over \dx x^k} \sum_{j=0}^n x^j
=
x^k {\dx^k\over \dx x^k} {1-x^{n+1}\over 1-x}
$$
e anche questo si presta al calcolo simbolico usando le primitive di
GiNaC, a quello che ho capito. 
Si noti che i coefficienti sono proprio quelli della forma del Lemma
qui sopra. In definitiva, 
se $f(n)=p(n)x^n$ (o una somma finita di quantit\`a dello stesso tipo)
con $x\ne1$, sappiamo 
trovare una formula chiusa per 
$$
\sum_{j=1}^n f(j).
$$
Il guaio \`e che se $x=1$ allora per calcolare la stessa quantit\`a
sono necessarie le formule 
di sommazione: in effetti esistono formule esatte quale che sia il
grado di $p$, ma sono piuttosto 
noiose da calcolare e non sono sicuro che esista un algoritmo generale
implementabile in un 
tempo ragionevole (si pu\`o guardare L. S. Levy, {\it Summation of the
series $1^n+2^n+\cdots+x^n$ 
using elementary calculus}, Amer. Math. Monthly {\bf 77} (1970),
840--847, {\it Corrigendum, ibidem}, 
{\bf 78} (1971), p. 987, di cui ho copia in ufficio). Comunque le
formule per i gradi fino a 4 
sono semplici. In generale so trovare la formula per il grado $d$
risolvendo un sistema lineare di 
$d+1$ equazioni: forse \`e il caso di scrivere l'algoritmo relativo
solo in un secondo tempo? 

Rapidamente, illustro qualche altro problema. Ho provato a risolvere
ricorrenze binarie non omogenee 
del tipo 
$$
x_n = \alpha x_{n-1} + \beta x_{n-2} + f(n)
\eqno(6)
$$
dove $f$ \`e un polinomio. Posto come prima 
$$
g_n = \alpha g_{n-1} + \beta g_{n-2}
$$
ma con la condizione iniziale $g_0=0$, $g_1=1$, ho trovato varie
formule per $x_n$, alcune mediante 
il procedimento illustrato sopra, mentre ne ho ottenute altre con
considerazioni dirette, e 
queste ultime hanno il vantaggio di coinvolgere solo $\alpha$ e
$\beta$ (ma sono pi\'u complesse). 
Per dare un esempio della complessit\`a, se $f(n)=\gamma$ (cio\`e \`e
costante) ho trovato queste 
due formule per $x_n$ ($n\ge1$): 
$$
x_n = x_1 g_n + \beta x_0 g_{n-1} + 
\gamma \sum_{k=0}^{[(n-1)/2]} \beta^k \sum_{j=0}^{2n-2k-2} {j+k\choose
j} \alpha^j
$$
(non garantisco che l'estremo superiore della prima somma sia
corretto) oppure 
$$
x_n = x_1 g_n + \beta x_0 g_{n-1} + 
\gamma \sum_{k=0}^{n-1} g_k
$$
e questa somma si calcola esplicitamente usando il fatto che $g_k$ \`e
combinazione lineare 
di potenze delle soluzioni dell'equazione caratteristica, ed usando di
nuovo la (5). 
In questo caso, bisogner\`a poi distinguere se $\alpha+\beta=1$
(cio\`e se $\lambda=1$ \`e 
soluzione dell'equazione caratteristica), eccetera. Si noti che nella
prima formula compaiono 
solo $\alpha$ e $\beta$ (e quindi, se questi sono numeri interi, nelle
formule si lavora solo 
con interi), ma contiene circa $n^2$ addendi, mentre, nel secondo
caso, alla fine del calcolo, 
si ha una formula che contiene solo le potenze $n$-esime delle
soluzioni dell'equazione 
caratteristica, che per\`o possono essere irrazionali o complesse. 

Pi\'u in generale, ho pi\'u o meno capito come risolvere la (6) se $f$
\`e un polinomio: come 
sempre, il trucco \`e sommare la ``parte omogenea'' $x_1 g_n + \beta
x_0 g_{n-1}$ ad una 
soluzione particolare della non omogenea, che se $f$ \`e un polinomio,
allora \`e un polinomio 
dello stesso grado di cui si devono determinare i coefficienti. Questo
conduce, ancora una volta, 
alla soluzione di un sistema lineare con $d+1$ incognite, se $d$ \`e
il grado di $f$. 
Tutto ci\`o se $\lambda=1$ non \`e soluzione dell'equazione
caratteristica: se \`e una 
soluzione di molteplicit\`a $m$, allora si deve cercare come soluzione
un polinomio di grado $d+m$. 

Esempi istruttivi (mi auguro): ometto i calcoli per brevit\`a e
perch\'e non ho voglia di copiarli, 
ma li ho fatti e posso mostrarveli quando serve. 
Qui $g$ \`e la soluzione dell'omogenea associata con le condizioni
iniziali $g_0:=0$, $g_1:=1$. 
Alcune soluzioni, cos\'\i\ come sono, sono valide solo per $n\ge1$:
diventano valide per $n\ge0$ 
se si pone (come \`e ragionevole) $g_{-1}:=1$. 

\begin{itemize}
\item 
$$
x_n = x_{n-1} + x_{n-2} + 1.
$$
Soluzione: qui $g_n=f_n$, l'$n$-esimo numero di Fibonacci. 
$$
x_n = x_1 f_n + x_0 f_{n-1} + (f_0 + f_1 + f_2 + \cdots + f_{n-1})
    = x_1 f_n + x_0 f_{n-1} + f_{n+1} - 1.
$$
Questo esempio \`e interessante perch\'e conosco un problema
combinatorio coi grafi che d\`a 
origine precisamente a questa successione con $x_0=0$, $x_1=1$. 

\item 
$$
\left\{
\begin{array}{l}
x_n = x_{n-1} + x_{n-2} + n \\
x_0 = 0, x_1 = 0            \\
\end{array}
\right.
$$
Soluzione: posto $q(n)=n+3$ ho 
$$
x_n = q(1) f_n + q(0) f_{n-1} - q(n) = 4f_n + 3f_{n-1} - n - 3.
$$

\item 
$$
\left\{
\begin{array}{l}
x_n = x_{n-1} + x_{n-2} + n^2 \\
x_0 = 0, x_1 = 0              \\
\end{array}
\right.
$$
Soluzione: posto $q(n)=n^2+6n+13$ ho 
$$
x_n = q(1) f_n + q(0) f_{n-1} - q(n) = 20 f_n + 13 f_{n-1} - n^2 - 6n
- 13.
$$
\item 
$$
\left\{
\begin{array}{l}
x_n = x_{n-1} + x_{n-2} + n^3 \\
x_0 = 0, x_1 = 0              \\
\end{array}
\right.
$$
Soluzione: posto $q(n)=n^3+9n^2+30n+54$ ho 
$$
x_n = q(1) f_n + q(0) f_{n-1} - q(n) = 94 f_n + 54 f_{n-1} - n^3 -
9n^2 - 30n - 54.
$$
In questi ultimi tre esempi la soluzione \`e sempre del tipo $x_n=q(1)
f_n + q(0) f_{n-1} - q(n)$, 
dove si deve determinare $q(n)$ in funzione del polinomio $p(n)$ che
\`e la parte non omogenea 
della ricorrenza. (Anche il primo esempio pu\`o essere trattato in
modo analogo). 
$q$ si determina imponendo che abbia lo stesso grado di $p$ e che
soddisfi l'equazione funzionale 
$$
q(n) = q(n-1) + q(n-2) - p(n).
\eqno(7)
$$
(N. B. Qui \`e essenziale che l'equazione caratteristica non abbia la
soluzione $\lambda=1$.) 
Infatti, se $q$ soddisfa (7), si ha 
$$
\begin{array}{rl}
x_0 &= q(1) f_0 + q(0) f_{-1} - q(0) = 0 \\
x_1 &= q(1) f_1 + q(0) f_0    - q(1) = 0 \\
x_n &= x_{n-1} + x_{n-2} + p(n) \cr
    &=            \( q(1) f_{n-1} + q(0) f_{n-2} - q(n-1) \) + \\
    &\qquad\qquad \( q(1) f_{n-2} + q(0) f_{n-3} - q(n-2) \) + p(n) \\
    &= q(1) \( f_{n-1} + f_{n-2} \) + q(0) \( f_{n-2} + f_{n-3} \) +
\\
    &\qquad\qquad p(n) - q(n-1) - q(n-2) \\
    &= q(1) f_n + q(0) f_{n-1} - q(n). \\
\end{array}
$$
Anche in questo caso, per determinare $q$ \`e necessario risolvere un
sistema lineare di $d+1$ 
equazioni, dove $d$ \`e il grado di $p$. Per esempio, se $p(n)=n^2$
devo trovare $a$, $b$, $c$ 
tali che 
$$
an^2+bn+c = \(a(n-1)^2 + b(n-1) + c\) + \(a(n-2)^2 + b(n-2) + c\) -
n^2
$$
cio\`e, semplificando 
$$
(a-1) n^2 + (b-6a) n + (5a - 3b + c) = 0
$$
da cui $a=1$, $b=6$, $c=13$. Dagli esempi che ho provato sembra
ragionevole supporre che, in 
generale, la matrice del sistema risultante sia triangolare, che
sarebbe una cosa buona. 

\item 
$$
\left\{
\begin{array}{l}
x_n = {1\over 2} x_{n-1} + {1\over 2} x_{n-2} + 1 \\
x_0 = 0, x_1 = 0                                  \\
\end{array}
\right.
$$
Qui c'\`e la radice $\lambda=1$ e $g(n)={2\over 3} \( 1 - ({-1\over
2})^n \)$. 
$$
x_n = {2\over 3} n - {4\over 9} \left( 1 - \left(-1\over2\right)^n
\right).
$$

\item 
$$
\left\{
\begin{array}{l}
x_n = 2 x_{n-1} - x_{n-2} + 1 \\
x_0 = 0, x_1 = 0              \\
\end{array}
\right.
$$
Qui la radice $\lambda=1$ \`e doppia e $g(n) = n$. 
$$
x_n = {n(n-1)\over 2}.
$$
\end{itemize}

Una teoria analoga vale per le ricorrenze lineare di ordine finito a
coefficienti costanti: 
evidentemente le cose sono pi\'u incasinate, ma la sostanza \`e la
stessa. 

\bigskip
\leftline{\bf Ricorrenze lineari di ordine 1 a coefficienti non
costanti}
\bigskip\noindent
Questo \`e in gran parte stato trattato nel messaggio del 14.12: \`e
sostanzialmente pi\'u 
difficile del caso precedente. Come ho gi\`a detto, si tratta di
equazioni del tipo 
$$
x_n = f(n) x_{n-1} + g(n).
$$
Anche qui basta trovare la soluzione generale dell'equazione omogenea
associata e sommare una 
qualsiasi soluzione dell'equazione completa, ma determinare
quest'ultima pu\`o essere complicato. 

\bigskip
\leftline{\bf Ricorrenze lineari non di ordine finito}
\bigskip\noindent
In queste rientrano quella del quicksort, che come detto si riconduce
ad una ricorrenza di ordine 
1 a coefficienti non costanti. Nella stessa categoria rientra quella
del problema combinatorio 
delle permutazioni senza punto fisso discussa nel mio messaggio
precedente, ed in questo caso 
non so dire molto in generale. 

\bigskip
\leftline{\bf Ricorrenze non lineari}
\bigskip\noindent
Sono ricorrenze molto complicate: tipico esempio (Lueker) $x_n =
3x_{n-1}^2$. Qui non conosco una 
teoria generale: se esiste probabilmente \`e molto complicata. La mia
domanda \`e la solita: 
quanto \`e rilevante questo caso? Nel caso qui sopra c\`e il trucco
(basta porre $y_n:=\log x_n$ 
ed allora come per magia $y_n=2y_{n-1}+\log 3$ \`e lineare a
coefficienti costanti), ma basta 
cambiare appena la forma ($x_n = 3x_{n-1}^2+1$) e siamo del gatto. 

\bigskip
\leftline{\bf ``Equazioni funzionali''}
\bigskip\noindent
Rientrano in questa categoria le ricorrenze tipo 
$$
x_n = 2 x_{n/2} + f(n)
$$
e simili, con l'opportuna interpretazione di $n/2$. Non so se si
possono considerare vere e proprie 
relazioni di ricorrenza (per questo uso un nome diverso): pi\'u che
altro esprimono una relazione 
tra i valori che la funzione assume su vari valori dell'argomento. Mi
\`e chiaro che sono molto 
interessanti per la tesi di Tatiana, ma a parte il ``Master Theorem''
qui non so dire molto. 
\`E chiaro che se non sono troppo complicate \`e possibile ricondurle
a relazioni di ricorrenza 
(vedi sempre Lueker): nel caso qui sopra se $n=2^m$, posto
$y_m=x(2^m)$ la relazione diventa
$$
y_m = 2 y_{m-1} + f(2^m),
$$
che \`e lineare del primo ordine a coefficienti costanti. Se $f$ non
\`e troppo complicata, pu\`o 
darsi che quest'ultima ricorrenza si possa trattare come quelle viste
sopra. Segnalo il fatto 
che per\`o stiamo considerando una sottosuccessione di $x_n$ piuttosto
poco densa (per questo 
ho spesso insistito sulla monotonia). Qui (come al punto precedente)
penso che si tratti di 
capire quale tipo di ricorrenze possono ragionevolmente capitare,
sempre ammesso che si possa, 
e studiare o inventare la teoria relativa. 

\bigskip
\centerline{Appendice: formule di sommazione} 
\bigskip
\noindent
Per comodit\`a, riporto senza dimostrazione le pi\'u importanti
formule di sommazione. 
Ne conosco anche altre, comode nel caso in cui si sappia per esempio
che $f$ \`e monot\`ona. 

\noindent{\bf Teorema (Formula di sommazione parziale di Abel). }{\sl
Sia $(a_n)_{n\in\N}$ una successione di numeri complessi e 
$\phi\in{\cal C}^1\(\R^{0+},\C)$ una qualsiasi funzione derivabile. 
Posto 
$$
A(x) :=  \sum_{n\le x} a_n, 
$$
per $x\ge1$ si ha 
$$
\sum_{n\le x} a_n \phi(n) = A(x)\phi(x) - \int_1^x A(t) \phi'(t) \,
\dx t.
$$}

\noindent{\bf Teorema (Formula di Sommazione di Euler-McLaurin). }{\sl
Sia $f\colon(x,y]\to\C$ una qualunque funzione derivabile. 
Si ha 
$$
\sum_{x<n\le y} f(n) = \int_x^y f(t) \, \dx t + \int_x^y \{t\} f'(t)
\, \dx t
- \{y\}f(y) + \{x\} f(x),
$$
dove $\{x\}$ indica la parte frazionaria del numero reale $x$.}

\`E del tutto evidente che queste sono utilizzabili nella pratica solo
se siamo in grado di 
calcolare esplicitamente delle primitive: segnalo il fatto che la
prima delle due si applica 
spesso con $a(n)=1$ e quindi si riduce a 
$$
\sum_{n\le x} \phi(n) = [x]\phi(x) - \int_1^x [t] \phi'(t) \, \dx t,
$$
dove $[t]$ indica la parte intera di $t$. N. B. Le due formule sono
sostanzialmente equivalenti, 
ma in pratica una delle due potrebbe essere pi\`u comoda da applicare.

\end{document}
