\documentclass{article}[12]
\textwidth=16.0cm
\textheight=25cm
\topmargin=-1.5cm
\oddsidemargin=1cm
\evensidemargin=1cm

\begin{document}

\def\dx{{\rm d}}
\def\N{\mathbf N}
\def\Z{\mathbf Z}
\def\R{\mathbf R}
\def\C{\mathbf C}
\def\({\bigl(}
\def\){\bigr)}
\def\dsize{\displaystyle}

\bigskip
\leftline{\bf Osservazioni varie}
\bigskip\noindent
\begin{itemize}
\item Un esempio divertente di ricorrenza di ordine 2 la cui equazione
caratteristica ha radici 
complesse \`e il seguente: 
$$
x_n = 2 x_{n-1} - 2 x_{n-2}, \quad x_0=0, \quad x_1=1;
$$
i primi valori sono 
$$
0,\, 1,\, 2,\, 2,\, 0,\, -4,\, -8,\, -8,\, 0,\, 16,\, 32,\, 32,\, 0,\,
\dots
$$
La formula generale d\`a 
$$
x_n = \frac1 {2i} \( (1+i)^n - (1-i)^n \).
$$
Dato che $1+i=\sqrt2 \cdot e^{i\pi/4}$, $1-i=\sqrt2 \cdot
e^{-i\pi/4}$, si pu\`o anche scrivere 
$$
x_n = -i2^{n/2-1} \( e^{in\pi/4} - e^{-in\pi/4}\)
    = 2^{n/2} \sin\frac{n\pi}4.
$$
\item Un esempio di ordine 3 \`e invece 
$$
x_n = 2 x_{n-1} - x_{n-2} + 2 x_{n-3},\quad x_0=a,\quad x_1=b,\quad
x_2=c.
$$
Qui l'equazione caratteristica ha le radici $\lambda_1=2$,
$\lambda_2=i$, $\lambda_3=-i$, e quindi 
la soluzione \`e del tipo 
$$
x_n = \alpha \cdot 2^n + \beta \cdot i^n + \gamma\cdot(-i)^n.
$$
Facendo pazientemente tutti i calcoli si trovano i valori di
$\alpha$, $\beta$ e $\gamma$ e in 
definitiva 
$$
x_n = \frac15 (a+c) 2^n + O_{a,b,c}(1)
$$
dove, volendo, \`e possibile dire esplicitamente come \`e fatta la
funzione nascosta dentro 
$O(1)$. 
\item Sembra necessaria una procedura per risolvere i sistemi lineari,
dato che questa serve 
anche per trovare la soluzione particolare delle ricorrenze di ordine
$\ge2$. 
\item Pu\`o essere utile consultare anche l'Enciclopedia on-line delle
successioni di interi 
all'indirizzo 
\leftline{\tt http:/$\!$/www.research.att.com/$\sim$njas/sequences/ }
Il responsabile si chiama Sloane, ed \`e citato in uno degli articoli
sulle ricorrenze. 

\item \`E possibile o utile definire un tipo Numero\_Reale che ha
un'espressione simbolica ed anche 
un'espressione reale o complessa approssimata? L'idea sarebbe quella
di usare l'espressione 
simbolica fin quando possibile e poi quella approssimata quando
necessario. Cerco di spiegarmi: 
invento all'uopo una notazione che potrebbe essere ridicola o (vel)
insensata, ma spero che 
si capisca lo stesso. Vorrei poter avere un oggetto che chiamo {\tt
phi} con due parti, la prima 
{\tt phi.symb} che vale $(1+\sqrt(5)/2$, la seconda {\tt phi.val} che
vale $1.6180339887499\dots$. 
In questo modo, se avessi un altro oggetto {\tt lambda} tale che {\tt
lambda.symb} vale 
$(1-\sqrt(5)/2$ e quindi {\tt lambda.val} vale
$-0.6180339887499\dots$, detto {\tt mi} l'oggetto 
prodotto di due, vorrei che {\tt mi.symb} valga $-1$ (esattamente),
mentre {\tt mi.val} potrebbe 
anche valere $-1.000000000001$. Mi sono spiegato? Ha un qualche minimo
di senso o di utilit\`a? 
E poi, \`e realizzabile? (NB Mathematica pi\'u o meno ha questa
opzione: se si definisce 
$x=(1+\sqrt(5)/2$, si ottiene il valore numerico approssimato solo se
si chiede $N[x]$, altrimenti 
restituisce il valore simbolico). Se si desidera che il risolutore
simbolico di equazioni alle 
differenze sia utilizzabile autonomamente, mi sembra utile un'opzione
del genere. 

\item Pu\`o avere senso scrivere una procedura che risolva
approssimativam ente le equazioni di 
grado $\ge3$ dato che il comportamento della successione ricorrente
\`e dominato dalla radice 
pi\'u grande in modulo? 

\item \`E possibile chiedere a GiNaC di semplificare un'espressione in
modo da raccogliere i pezzi 
come ci fa pi\'u comodo? In altre parole, si pu\`o ottenere la
seguente uguaglianza? 
$$
n\cdot 2^n + n - 3 + 3^n -n^2\cdot2^n
=
3^n + (-n^2+n)\cdot 2^n + (n-3).
$$
Il senso della domanda \`e questo: sappiamo calcolare la somma
dell'espressione a sinistra 
per $n=0$,\dots, $N$ con l'algoritmo di cui ho gi\`a parlato applicato
a ciascun termine 
singolarmente, ma \`e ovviamente pi\'u efficiente applicare lo stesso
algoritmo ai tre termini 
a destra. 
\end{itemize}

\bigskip
\leftline{\bf Ricorrenze lineari di ordine 2 a coefficienti costanti
non omogenee}
\bigskip\noindent
Per completare il discorso del messaggio del 24.12.2001, se vogliamo
risolvere 
$$
x_n = \alpha x_{n-1} + \beta x_{n-2} + f(n),
\eqno(1)
$$
dove $f$ \`e una funzione definita sui numeri naturali (in sostanza
\`e la parte non omogenea 
dell'equazione alle differenze), consideriamo come sopra la
successione 
$$
g_n = \alpha g_{n-1} + \beta g_{n-2}
$$
con la condizione iniziale $g_0=0$, $g_1=1$, che si risolve
semplicemente mediante l'equazione 
caratt eristica. Non \`e difficile dimostrare per induzione che la
soluzione di (1) \`e 
$$
x_n = g_n x_1 + \beta g_{n-1} x_0 + \sum_{k=2}^n g_{n+1-k} f(k)
$$
per $n\ge1$. Dato che $g_n$ ha l'espressione esplicita 
$$
g_n = {\lambda_1^n - \lambda_2^n\over \lambda_1-\lambda_2}
$$
(suppongo tacitamente $\lambda_1\ne\lambda_2$, cio\`e
$\alpha^2+4\beta\ne0$, altrimenti la 
formula \`e leggermente diversa ma il principio \`e lo stesso),
abbiamo 
$$
\begin{array}{rl}
x_n &= g_n x_1 + \beta g_{n-1} x_0 +
\dsize{\sum_{k =2}^n {\lambda_1^{n+1-k} - \lambda_2^{n+1-k}\over
\lambda_1-\lambda_2} f(k) }\\
    &= g_n x_1 + \beta g_{n-1} x_0 +\dsize{
{\lambda_1^{n+1}\over \lambda_1-\lambda_2}\sum_{k=2}^n \lambda_1^{-k}
f(k) -
{\lambda_2^{n+1}\over \lambda_1-\lambda_2}\sum_{k=2}^n \lambda_2^{-k}
f(k) }.\\
\end{array}
$$
Se $f$ \`e un polinomio o una combinazione lineare di polinomi per
esponenziali, le somme 
qui sopra sono ancora combinazioni lineari di polinomi per
esponenziali, e abbiamo gi\`a 
visto che si possono sommare espl icitamente (intendo dire che c'\`e
un algoritmo per calcolare 
la formula chiusa). Un procedimento simile, anche se pi\'u complicato,
vale per le ricorrenze 
lineari di ordine finito a coefficienti costanti non omogenee,
qualunque sia l'ordine. 

\bigskip
\centerline{Altre formule chiuse} 
\bigskip
\noindent
Per completezza, aggiungo senza dimostrazione qualche altra formula
chiusa, presumibilmente 
non molto interessante ai fini dell'analisi dei programmi: per
$\theta\notin2\pi\Z$ 
$$
\begin{array}{rl}
\dsize{
\sum_{k=0}^N \sin(k\theta)} &= \dsize{{\cos(\theta/2) -
\cos((N+1)\theta/2)\over\sin(\theta/2)} }\\
\dsize{
\sum_{k=0}^N \cos(k\theta)} &= \dsize{\frac12 +       
{\sin((N+1)\theta/2)\over\sin(\theta/2)} }\\
\end{array}
$$
Da ciascuna di queste segue che le due somme in questione, prese in
valore assoluto, sono 
$$
\le \min\(N+1,|\sin(\theta/2)|\) = O\( \min (N+1,
\Vert\theta/\pi\Vert^{-1} ) \)
$$
dove $\Vert x\Vert$ indica la distanza di $x$ dall'intero pi\'u vicino

($\Vert x\Vert=\min(\{x\},1-\{x\})$). 

\bigskip
\centerline{Appendice: un'altra formula di sommazione} 
\bigskip
\noindent
\noindent{\bf Teorema. }{\sl
Sia $f\colon\R^+\to\R^+$ una funzione debolmente decrescente e
infinitesima. 
Esiste una costante reale $E$ tale che per $x\to\infty$ si ha 
$$
\sum_{n\le x} f(n) = \int_1^x f(t)\,\dx t + E + O\(f(x)\).
$$
In effetti 
$$
0\le E = \sum_{n\ge1} \left\{f(n)-\int_n^{n+1} f(t)\,\dx t\right\} \le
f(1).
$$} 

\end{document}
